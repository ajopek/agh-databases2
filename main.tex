\documentclass[12pt]{article}
  \usepackage{geometry}
  \geometry{
    a4paper,
    total={170mm,257mm},
    left=20mm,
    top=20mm,
  }
%Packages
\usepackage{polski}
\usepackage[T1]{fontenc}
\usepackage[utf8]{inputenc}
\usepackage{color}   %May be necessary if you want to color links
\usepackage{listings}
\usepackage{graphicx}
\usepackage{float}
\usepackage[hidelinks,linktoc=all]{hyperref}
\usepackage{hyperref}

%listings config
\definecolor{mygreen}{rgb}{0,0.6,0}
\definecolor{mygray}{rgb}{0.5,0.5,0.5}
\definecolor{mymauve}{rgb}{0.58,0,0.82}

\lstset{ %
  backgroundcolor=\color{white},   % choose the background color; you must add \usepackage{color} or \usepackage{xcolor}; should come as last argument
  basicstyle=\footnotesize,        % the size of the fonts that are used for the code
  breakatwhitespace=false,         % sets if automatic breaks should only happen at whitespace
  breaklines=true,                 % sets automatic line breaking
  captionpos=b,                    % sets the caption-position to bottom
  commentstyle=\color{mygreen},    % comment style
  deletekeywords={...},            % if you want to delete keywords from the given language
  escapeinside={\%*}{*)},          % if you want to add LaTeX within your code
  extendedchars=true,              % lets you use non-ASCII characters; for 8-bits encodings only, does not work with UTF-8
  frame=single,	                   % adds a frame around the code
  keepspaces=true,                 % keeps spaces in text, useful for keeping indentation of code (possibly needs columns=flexible)
  keywordstyle=\color{blue},       % keyword style
  language=SQL,                 % the language of the code
  morekeywords={*,...},            % if you want to add more keywords to the set
  numbers=left,                    % where to put the line-numbers; possible values are (none, left, right)
  numbersep=5pt,                   % how far the line-numbers are from the code
  numberstyle=\normalsize\color{mygreen}, % the style that is used for the line-numbers
  rulecolor=\color{black},         % if not set, the frame-color may be changed on line-breaks within not-black text (e.g. comments (green here))
  showspaces=false,                % show spaces everywhere adding particular underscores; it overrides 'showstringspaces'
  showstringspaces=false,          % underline spaces within strings only
  showtabs=false,                  % show tabs within strings adding particular underscores
  stepnumber=1,                    % the step between two line-numbers. If it's 1, each line will be numbered
  stringstyle=\color{mymauve},     % string literal style
  tabsize=2	                   % sets default tabsize to 2 spaces                   % show the filename of files included with \lstinputlisting; also try caption instead of title
}
%Define more keywords
\lstdefinelanguage[Transact]{SQL}[]{SQL}{
  morekeywords={FUNCTION, RETURNS, RETURN, BEGIN, TRY, CATCH, PROCEDURE, TRIGGER, VIEW},
}

\lstset{language=[Transact]SQL,
       }

%Content
\begin{document}

\title{Bazy Danych 2:\\Lab 1}
\author{Artur Jopek}
\date{18.10.2018}
\maketitle


\maketitle

\section{Widoki}
\subsection{}
wycieczki\_osoby(kraj,data, nazwa\_wycieczki, imie, nazwisko,status\_rezerwacji)
\lstinputlisting[firstline=1, lastline=11]{sql/lab1/widoki.sql}
\subsection{}
wycieczki\_osoby\_potwierdzone (kraj,data, nazwa\_wycieczki, imie,nazwisko,status\_rezerwacji)
\lstinputlisting[firstline=14, lastline=23]{sql/lab1/widoki.sql}
\subsection{}
wycieczki\_przyszle (kraj,data, nazwa\_wycieczki, imie, nazwisko,status\_rezerwacji)
\lstinputlisting[firstline=26, lastline=37]{sql/lab1/widoki.sql}
\subsection{}
wycieczki\_miejsca(kraj,data, nazwa\_wycieczki,liczba\_miejsc, liczba\_wolnych\_miejsc)
\lstinputlisting[firstline=40, lastline=49]{sql/lab1/widoki.sql}
\subsection{}
dostępne\_wyciezki(kraj,data, nazwa\_wycieczki,liczba\_miejsc, liczba\_wolnych\_miejsc)
\lstinputlisting[firstline=60, lastline=68]{sql/lab1/widoki.sql}
\subsection{}
rezerwacje\_do\_anulowania – lista niepotwierdzonych rezerwacji które powinne zostać
anulowane, rezerwacje przygotowywane są do anulowania na tydzień przed wyjazdem)
\lstinputlisting[firstline=52, lastline=57]{sql/lab1/widoki.sql}

\section{Procedury}

%\lstinputlisting{firstline=1, lastline=10}{sql/lab1/procedury.sql}

\section{Triggery}

%\lstinputlisting{firstline=1, lastline=10}{sql/lab1/triggery.sql}

\end{document}
